\documentclass[12pt, letterpaper, oneside]{book}
\usepackage{amsmath}
\usepackage[skip=6pt]{parskip}
\begin{document}
\chapter{Algebra Preliminaries}
\section*{Credit}
This entire chapter with almost all of the examples comes from Paul's Online Math Notes found here:

https://tutorial.math.lamar.edu/Classes/Alg/Alg.aspx

\section{Integer Exponents}
To begin, only looking at positive integers, the definition of exponentiation is below.
\begin{equation}
    a^n = \underbrace{ a \cdot a \cdot  a\cdots a}_\textrm{n times}, \quad \text{where } n > 0
\end{equation}

For example, $ 3^3 = 3 \cdot 3 \cdot 3 \cdot 3 = 27 $

One idea to note is the importance of parenthesis. If no parenthesis is present around the base, then it is the same as multiplying by -1.
\[-2^4 \neq (-2)^4 \]
\[(-2)^4 = -2 \cdot -2 \cdot -2 \cdot -2 = 16\] 
\[-2^4 = -1 \cdot (2^4) = -1 \cdot (2 \cdot 2 \cdot 2 \cdot 2) = -16\]
From here you can see that when the exponent is an odd number the result is always a negative number.

One identity of exponents is the idea of 0 being the exponent.
\begin{equation}
    a^0 = 1, \quad \text{where } a \neq 0
\end{equation}
It does not matter if a is positive or negative, the result is always 1.

Now lets look at the last case for integers, where they are negative.
\begin{equation}
    a^{-n} = \frac{1}{a^n}
\end{equation}
For example, $(-4)^{-3} = -\frac{1}{4} \cdot -\frac{1}{4} \cdot -\frac{1}{4} = -\frac{1}{64}$

Moving forward there are a lot of properties of exponents that need to be covered. These properties only apply when the base is the same for all cases.
\begin{equation}
    a^n a^m = a^{n + m}
\end{equation}
\begin{equation}
    (a^n)^m = a^{n \cdot m}
\end{equation}
\begin{equation}
    \frac{a^n}{a^m} = a^{n - m} = \frac{1}{a^{n-m}} \quad \text{where } a \neq 0
\end{equation}
\begin{equation}
    \frac{1}{a^{-n}} = a^n
\end{equation}
These next properties apply to problems where the base of the exponents are different
\begin{equation}
    (ab)^n = a^n b^n
\end{equation}
\begin{equation}
    \left(\frac{a}{b}\right)^n = \frac{a^n}{b^n} \quad \text{where } b \neq 0
\end{equation}
\begin{equation}
    \left(\frac{a}{b}\right)^{-n} = \left(\frac{b}{a}\right)^n = \frac{b^n}{a^n} 
\end{equation}
\begin{equation}
    (ab)^{-n} = \frac{1}{(ab)^n} 
\end{equation}
\begin{equation}
    \frac{a^{-n}}{b^{-m}} = \frac{b^m}{a^n} 
\end{equation}
\begin{equation}
    (a^nb^m)^k = a^{nk}b^{mk} 
\end{equation}
\begin{equation}
    \left(\frac{a^n}{b^m}\right)^k = \frac{a^{nk}}{b^{mk}}
\end{equation}

\subsection{Practice Problem 1}
Simplify the following:
\[ (-10z^2y^{-4})^2(z^3y)^{-5} \]
Bring in overarching exponents
\[ (-100z^4y^{-8})(z^{-15}y{-5}) \]
Write as a fraction, brining negative exponents to denominator
\[ \frac{-100z^4}{y^8y^5z^{15}} \]
Add / Subtract exponents
\[ \frac{-100}{y^{13}z^{11}} \]



\subsection{Practice Problem 2}
Simplify the following:
\[ \left(\frac{24a^3b^{-8}}{6a^{-5}b}\right)^{-2} \]
First simplify everything within the parenthesis. Bring the larger negative exponents to the top / bottom
\[ \left(\frac{4a^3a^5}{bb^8}\right)^{-2} \]
Simplify by adding exponents
\[ \left(\frac{4a^8}{b^9}\right)^{-2} \]
Make overarching exponent positive
\[ \left(\frac{b^9}{4a^8}\right)^{2} \]
Bring in overarching exponent to numerator and denominator
\[ \frac{b^{18}}{16a^{16}} \]

\section{Rational Exponents}
Now we are going to deal with exponents that are not just integers. An example of a rational exponent is
\[b^{\frac{m}{n}}\]
Starting with a simple case where $m = 1$
\[a = b^{\frac{1}{n}} \rightarrow a^n = b\]
This is called the n-th root of $b$. You've probably seen the exponent $a^{\frac{1}{2}}$, which is the same as $\sqrt{a}$ . A way of looking at these exponents is what base can I raise the inverse of the fraction in order to get the same base from the original problem.

For example:
\[81^{\frac{1}{4}}\]
Simplify to lamens terms.
\[?^{4} = 81\]
Solve
\[3^4 = 81 \rightarrow 81^{\frac{1}{4}} = 3\]

Going to more complicated situations where $m > 1$ .
It is important to know that all the same properties from the section before still apply and will come in handy.

For example:
\[8^{\frac{2}{3}}\]
Bring out the squared numerator. Remember the equation in $(a^n)^m = a^{n \cdot m}$, but in reverse
\[ 8^{\frac{2}{3}} = \left( 8^{\frac{1}{3}} \right)^2 \]
Solve the inner parenthesis
\[ 8^{\frac{1}{3}} \rightarrow ?^3 = 8 \rightarrow 2^3 = 8\]
\[ 8^{\frac{1}{3}} = 2\]
Bring back into original problem
\[\left( 8^{\frac{1}{3}} \right)^2 = 2^2 = 4\]

\subsection{Practice Problem 1}
Solve:
\[\left( \frac{w^{-2}}{16v^{\frac{1}{2}}} \right) ^ {\frac{1}{4}}\]
Move negative exponents
\[\left( \frac{1}{16w^{2}v^{\frac{1}{2}}} \right) ^ {\frac{1}{4}}\]
Distribute overarching exponent
\[\frac{1}{2w^{\frac{1}{2}}v^{\frac{1}{8}}}\]

\section{Radicals}
A radical is defined as below:
\begin{equation}
    \sqrt[n]{a} = a^{\frac{1}{n}} \quad \text{where } n > 0
\end{equation}
Also as a note $n$ also has to be an integer. As you can see this is no different than the Integer Exponents section, and all the same rules apply.

One note is the square root where
\[ \sqrt[2]{a} = \sqrt{a} \]

One common mistake is that the exponent is distrubuted amongst all bases. That case is only true if the base is also grouped with the exopnent. Shown below is an example.
\[8x^{\frac{1}{10}} =8\sqrt[10]{x} \neq \sqrt[10]{8x} \]

Another very important rule is neative bases. We know that $\sqrt{16} = 4$. However, $(-4)^2 = 16$.

In addition to this section is the general rational exponent notation, which is as follows:
\begin{equation}
    a^{\frac{m}{n}} = (a^m)^{\frac{1}{n}} = \sqrt[n]{a^m}
\end{equation}

With the general notation there are also a few rules. These rules apply where $n$ is a positive integer, $n >1$, and $a$ and $b$ are positive real numbers.
\begin{equation}
    \sqrt[n]{a^n} = a
\end{equation}
\begin{equation}
    \sqrt[n]{ab} = \sqrt[n]{a}\sqrt[n]{b}
\end{equation}
\begin{equation}
    \sqrt[n]{\frac{a}{b}} = \frac{\sqrt[n]{a}}{\sqrt[n]{b}}
\end{equation}

A very common mistake is with addition and subtraction of radicals. The following cannot be done on radicals.
\[\sqrt[n]{a \pm b} \neq \sqrt[n]{a} \pm \sqrt[n]{b} \]

\subsection{Practice Problem 1}
Write in the exponent form and solve:
\[ \sqrt[4]{16} \]
Solution:
\[ \sqrt[4]{16} = 16^{\frac{1}{4}} \]
\[?^4 = 16\]
\[2^4 = 16\]
\[ \sqrt[4]{16} = 2 \]

\subsection{Practice Problem 2}
Simplify the following:
\[\sqrt[4]{32 x^9 y^5 z^{12}}\]
Make all multiples of 4
\[ \sqrt[4]{16 x^8 y^4 z^{12} (2xy)} \]
Bring out multiples of 4
\[ \sqrt[4]{16} \cdot \sqrt[4]{(x^2)^4} \cdot \sqrt[4]{y^4} \cdot \sqrt[4]{z^{12}} \cdot \sqrt[4]{2xy}\]
Simplify
\[2x^2yz^3\sqrt[4]{2xy}\]

\section{Polynomials}
Polynomials are algebraic expressions in the form of $ax^n$ where the following criteria are true: $n$ is a non-negative integer, $a$ is a real number and is called the coefficient. The degree of a polynomial is the highest exponent within the polynomial.

Below are some examples of polynomials and their degrees.
\[2x^7 + 3x^3 + 2 \quad \text{degree: 7}\]
\[x^2 + 14x - 21 \quad \text{degree: 2}\]
\[4 \quad \text{degree: 0}\]

From the rules above the following would not meet the criteria of a polynomial
\[4x^6 + 15x^{-8} + 1\]
\[5\sqrt{x} - x + x^2\]
The first one is not a polynomial because it has a negative exponent. The second one is not a polynomial because you cannot factor out the radical. Note that only the exponent cannot be a radical, but the coefficient / base can be one.

Up to this point we have been dealing with polynomials of one variable. Polynomials of two variables come in the form $ax^ny^m$. The degree is now the sum of the exponents, $n+m$. So for example:
\[x^2y - 6x^3y^{12}+10x^2-7y+1 \quad \text{degree 15} \]
In reality there can be any number of terms for polynomials. A monomial refers to one variable, a biomial refers to two, and a trinomal refers to three.

Back to polynomials, we need to discuss adding and subtracting polynomials. The distrubitve law will come in handy
\begin{equation}
    a(b+c) = ab + ac
\end{equation}
One repeated pattern that comes out of this law is:
\begin{equation}
    (a+b)(a-b) = a^2 + b^2
\end{equation}
\begin{equation}
    (a+b)^2=a^2+2ab+b^2
\end{equation}
\begin{equation}
    (a-b)^2=a^2-2ab+b^2
\end{equation}
\subsection{Practice Problem 1}
Add $6x^5 - 10x^2 + x - 45$ to $13x^2 - 9x + 4$

Combine like terms:
\[6x^5 + (-10 + 13)x^2 + (-9 + 1)x + (-45 + 4)\]
Simplify:
\[6x^5 + 3x^2 -8x - 41\]

\subsection{Practice Problem 2}
Multiply the following:
\[(3x + 7y)(x-2y)\]
Multiply out the $3x$:
\[3x(x-2y) \rightarrow 3x^2 - 6xy \]
Multiply out the $7y$:
\[7y(x-2y) \rightarrow 7xy-14y^2 \]
Combine:
\[3x^2 - 6xy + 7xy-14y^2\]
Simplify:
\[3x^2 + xy - 14y^2\]

\section{Factoring Polynomials}
Factoring is the process of figuring out what was multiplied for a given solution. For example with 12, we could have multiplied $6\cdot2$ or $4\cdot3$. Even $\frac{1}{2}\cdot24$. Essentially this process is simplifying down a polynomial until nothing more can be done.

For example:
\[x^2 - 16 = (x+4)(x-4)\]
This is the polynomial in the mostfactored form. Nothing else can be done to simplify it down.

One of the first methods for factoring is taking out the greatest common factor. Basically the opposite of the distrubitve law.
\[2x^2+4z+12 \rightarrow 2(x^2+2x+6)\]

There are some situations where the factors follow the so called special forms (listed below).
\begin{equation}
    (a^2 + 2ab + b^2) = (a+b)^2
\end{equation}
\begin{equation}
    (a^2 - 2ab + b^2) = (a-b)^2
\end{equation}
\begin{equation}
    (a^2 - b^2) = (a+b)(a-b)
\end{equation}
\begin{equation}
    (a^3 + b^3) = (a+b)(a^2 - 2ab + b^2)
\end{equation}
\begin{equation}
    (a^3 - b^3) = (a-b)(a^2 + 2ab + b^2)
\end{equation}

\end{document}